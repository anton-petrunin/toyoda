\documentclass{article}
\usepackage{quad}

\begin{document}


\title{Five points in a CAT(0) space after  Tetsu Toyoda}
\author{Nina Lebedeva and Anton Petrunin}
%\address{Anton Petrunin, Math. Dept., PSU,University Park, PA 16802, USA.}\address{petrunin@math.psu.edu}
\date{}
\maketitle
\begin{abstract}
We give another proof of Toyoda's theorem that gives a nessesary and sufficient condition for the existance of a distance preserving map from a given 5-point metric space to a $\CAT(0)$ space. 
\end{abstract}

\section{Introduction}

The $\CAT(0)$ comparison is a certain inequality for 6 distances between 4 points in a metric space 
that can be defined the following way.

\begin{thm}{Definition}
A quadruple of points $(p_1,p_2,x_1,x_2)$ in a metric space $X$ satisfies $\CAT(0)$ comparison
if there is quadruple of points $(\~p_1,\~p_2,\~x_1,\~x_2)$ in the Euclidean space $\EE^3$ such that 
\begin{align*}
|p_1-p_2|_{X}&\le|\~p_1-\~p_2|_{\EE^3},
\\ 
|x_1-x_2|_{X}&\le|\~x_1-\~x_2|_{\EE^3},
\\ 
|p_i-x_j|_{X}&\ge|\~p_i-\~x_j|_{\EE^3},
\end{align*}
for any $i$ and $j$.

A metric space $X$ is called $\CAT(0)$ if every quadruple of points in $X$ satisfies $\CAT(0)$ comparison.
\end{thm}

The definition above is equivalent to the (2+2)-point comparison in \cite{alexander-kapovitch-petrunin}
where other definitions and their equivalence are discussed as well.
Note that we do not assume that $\CAT(0)$ is a length space;
for example any 3 point metric space is $\CAT(0)$.


It is not hard to check that if a quadruple of points satisfies $\CAT(0)$ comparison for all relabeling,
then it admits a distance-preserving inclusion into a length $\CAT(0)$ space.
The following theorem generalizes this statement to 5 point metric spaces.

\begin{thm}{Theorem}
Let $P$ be a 5 point metric space that satisfies $\CAT(0)$ comparison.
Then $P$ admits a distance-preserving inclusion into a $\CAT(0)$ length space $X$.

Moreover,
$X$ can be chosen to be a subcomplex of a 4-simplex such that (1) each simplex in $X$ has Euclidean metric and (2) the inclusion maps the 5 points on $P$ to the vertexes of the 4-simplex.
\end{thm}

This theorem was proved by Tetsu Toyoda \cite{toyoda};
we present a shorter and more conceptual proof of this statement.
The proof uses the fact that convex space-like hypersurfaces in $\RR^{3,1}$ equipped with the induced length metrics are $\CAT(0)$ spaces.
We construct a distance-preserving inclusion $\iota$ of $P$ into $\RR^4$ or $\RR^{3,1}$.
In the case of $\RR^4$ the convex hull of $\iota(P)$ can be taken as $X$;
in the case of $\RR^{3,1}$ we take as $X$ a certain convex space-like hypersurface in $\RR^{3,1}$.





\section{Convex surfaces in spacetime}

Consider the spacetime $\RR^{3,1}$;
let us denote by $C^+$ and $C^-$ its future and past cones;
that is
\[C^\pm=\set{(x,y,z,t)\in\RR^{3,1}}{t^2<x^2+y^2+z^2,\ 0<\pm t}.\]

Let $K$ be a convex body in $\RR^{3,1}$;
denote by $\Sigma$ the surface of $K$. 
We say that $p\in \Sigma$ lies on the upper side of $\Sigma$ (briefly $p\in\Sigma^+$) if there is a space-like hyperplane in $\RR^{3,1}$ that supports $\Sigma$ at $p$ from above;
equivalently if $p+C^+$ does not intersect $K$.

Similarly we define lower side of $\Sigma$ denoted by $\Sigma^-$.
Note that $\Sigma^+$ and $\Sigma^-$ might have common points.
The subsets $\Sigma^+$ and $\Sigma^-$ are weakly spacelike;
in particular the length of any Lipschitz curve in these subsets can be defined and it leads to induced intrinsic metrics on $\Sigma^+$ and $\Sigma^-$. 

\begin{thm}{Lemma}\label{lem:sides}
Let $\Sigma$ be the surface of a convex body $K$ in $\RR^{3,1}$.
Then its upper and lower sides $\Sigma^+$ and $\Sigma^-$ equipped with induced intrinsic metric are a $\CAT(0)$ length spaces. 
\end{thm}

\parit{Proof.}
Applying Gauss formula, we see that a smooth convex space-like hypersurface in $\RR^{3,1}$ has nonpositive curvature.
Therefore any complete smooth convex space-like hypersurface in $\RR^{3,1}$ is $\CAT(0)$.

Note that $K^-=K+C^+$ a convex set and its boundary $\partial K^-$ is weakly space-like.
The surface $\partial K^-$ contains $\Sigma^-$ as a subset.
Moreover the induced length pseudometric on $\partial K^-$ is isometric to $\Sigma^-$.
Indeed every point of $\partial K^-$ either belongs to  $\Sigma^-$, or lies on a light ray emerging from $\Sigma^-$.
The statement follows since light rays have zero length.

Observe that $\partial K^-$ can be approximated by smooth space-like convex hypersurfaces.
In particular any 4-point subspace in $\partial K^-$ can be approximated by 4 -point subspaces in $\CAT(0)$ space.
Whence the $\CAT(0)$ comparison holds for all quadruples.
\qeds




Assume $v$ is a nonzero vector in $\RR^{3,1}$ and $p\in\Sigma$.
We say that $p$ lies on upper side of $\Sigma$ with respect to $v$ (briefly $p\in \Sigma^+(v)$) if $p+t\cdot v\notin K$ for any $t>0$.
Correspondingly, $p$ lies on lower side of $\Sigma$ with respect to $v$ (briefly $p\in \Sigma^-(v)$) if $p+t\cdot v\notin K$ for any $t<0$.

Note that 
\[\Sigma^\pm=\bigcap_{v\in C^+}\Sigma^\pm(v).\leqno({*})\]

\section{Projections of simplex}

Consider one-parameter family of affine maps $f_t$ from 4-simplex $K$ to $\RR^3$.
Suppose that the for any $t$ the images of edges of the simplex intersect only at their common vertexes.
Let us write $+1$, $0$, or $-1$ at each facet of $K$,

Consider the sum $\#(t)$ of orientations of the 5 facets 


Consider 4 planes in $\RR^3$ in general position.
They divide the space into 14 open cones --- 4 of them are cones over open triangles and 10 over 

Consider 5 hyperplanes in general position in $\RR^4$;
they divide the space into open cones 

If $K$ is a simplex in $\RR^4$ and $\Sigma$ be its boundary.
Choose a nonzero vector $v$.
Let $\Pi$ be a hyperplane transveral to $v$;
if we fix a Euclidean metric on $\RR^4$, then we may assume that $\Pi$ is perpendicular to $v$.

A facet $F$ of $K$ will be called \emph{upper} (\emph{lower}) with respect to $v$ if its projection to $\Pi$ along $v$ is orientation preserving (respectively orientation reversing).
If a facet is neither upper or lower, then it is called \emph{vertical} with respect to $v$.
(We assume that $\Sigma$ and $\Pi$ come with a natural choice of orientations so if we choose $(x,y,z,t)$-coordinates with $t$-axis in the direction of $v$, then the definition will agree with its intuitive meaning.)


Assume $v$ is generic, that is $v$ is not parallel to any facet of $K$.
In this case $K$ has either $1$, $2$, $3$, or $4$ upper facets; denote this number by $\#(v)$.

Assume $w$ is another generic vector that is connected to $v$ by a one-parameter family of nonzero vectors $v_t$.
Suppose that $\#(v)\le 2$ and $\#(w)\le 2$ then for some time moment $t_0$,
$K$ has 2 upper, 2 lower and 1 vertical facet, or 

Further we will be interested in the case when $K$ is a 4-simplex in $\RR^{3,1}$.
Note that for each vector $v\ne 0$, there is at least one upper and one lower facet. 




\begin{thm}{Observation}
Suppose $\Sigma$ is the surface of a convex polyhedron $K$ in $\RR^{3,1}$ and $p\in\Sigma$.
Then $p\in\Sigma^\pm$ if for any $v\in C^+$ there is a facet $F\ni p$ of $K$ such that $F\subset \Sigma^\pm(v)$.
\end{thm}



Note that $p\in\Sigma^+$ if and only if $p$ lies on upper side of $\Sigma$ with respect to any future oriented time-like vector $v$.

Let $K$ be a simplex in $\RR^{3,1}$
and $v$ is a generic time-like vector;
that is $v$ is not parallel to any facet of $K$.
Then for each facet $F$ of $P$ is upper or lower with respect to $v$;
that is, for a point $x$ in the interior of $F$ the line $x+v\cdot t\notin K$ for $t>0$ or $t<0$ respectively.




\section{Associated form}

Let us construct a quadratic form $W_{\bm{x}}$ on $\RR^{n-1}$ 
for given $n$-point array $\bm{x}\z=(x_1,\dots,x_n)$ in a metric space $X$.

Fix a nondegenerate simplex $\triangle$ in $\RR^{n-1}$.
Denote by $v_1,\dots,v_n$ its vertices.
If $(e_1,\dots,e_{n-1})$ is the standard basis on $\RR^{n-1}$,
we may assume that $v_i=e_i$ for $i<n$ and $v_n=0$.

Let us denote by $|a-b|_X$ the distance between points $a$ and $b$ in the metric space $X$.
Set
\[W_{\bm{x}}(v_i-v_j)=|x_i-x_j|^2_X\] 
for all $i$ and $j$.
Note that this identity define $W_{\bm{x}}$ uniquely.


The constructed quadratic form $W_{\bm{x}}$ will be called \emph{form of the point array $\bm{x}$ with respect to the simplex $\triangle$}.

Note that an array $\bm{x}=(x_1,\dots,x_n)$ in a metric space $X$ is isometric to an array in Euclidean space if and only if 
$W_{\bm{x}}(v)\ge 0$
for any $v\in \RR^{n-1}$.

In particular,  the
condition $W_{\bm{x}}\ge 0$ for triples of points means that 
all three triangle inequalities hold.
The following statement is an equivalent reformulation:

\begin{thm}{Observation}\label{triangle-inq}
Suppose $L$ be a subspace of $\RR^{n-1}$ such that
$W_{\bm{x}}(v)< 0$ for any nonzero vector $v\in L$.
Then in the projections of any 3 vertexes of $\triangle$ to the quotient space $\RR^{n-1}/L$ form a nondegenerate triangle.
\end{thm}

\section{Alexandrov's 4-point comparison}

Now let us discuss the relation between form of quadruples
and geometry of the space.
In this case $\triangle$ is a tetrahedron on $\RR^3$.

From the $3$-point case, 
it follows that $W_{\bm{x}}$ 
is nonnegative on every plane parallel to a face of the tetrahedron $\triangle$.
In particular, $W_{\bm{x}}$ can have at most one negative eigenvalue.

Assume $W_{\bm{x}}(w)<0$ for some $w\in\RR^3$.
From \ref{triangle-inq},
$w$ is transversal to each of 4 planes parallel to a faces of $\triangle$.

Consider the projection of $\triangle$ along $w$ to a transversal plane. 
Note that in the projection the 4 vertices of $\triangle$ lie in general position; 
that is, no three of them lie on one line.
Therefore  we can see one of two combinatorial pictures shown on the diagram.
It is easy to see that the combinatorics of the picture does not depend on the choice of $w$.


+PIC

If we see the diagram on the left we say that $\bm{x}$ is 
of type $\quadra(4)$ and otherwise we say that it is of type $\quadra(3)$.


\begin{thm}{Observation}\label{triangle-inq}
Suppose a metric on $\bm{x}=(x_1,\dots,x_n)$ satisfies $\CAT(0)$ condition
and $W_{\bm{x}}$ is its asociated form on $\RR^{n-1}$.
Suppose $L$ be a subspace of $\RR^{n-1}$ such that
$W_{\bm{x}}(v)< 0$ for any nonzero vector $v\in L$.
Then if the projections of 4 vertexes of $\triangle$ to the quotient space $\RR^{n-1}/L$ lies in one plane, then one of the points lies in the triangle formed by the remaining three points.
\end{thm}

The following statements give a connection between the forms $W_{\bm{x}}$ of the quadruple $\bm{x}$
and the curvature bounds in the sense of Alexandrov.
Their proofs are left to the reader.

Assume $X$ is a complete space with intrinsic metric.
Then
\begin{itemize}
\item If $W_{\bm{x}}\ge 0$ 
for any quadrilateral $\bm{x}=(x_1,\dots,x_4)$ 
then $X$ is isometric to a closed convex set in a Hilbert space. 
\item $X$ has no quadruples of type $\quadra(3)$ if and only if 
$X$ has nonnegative curvature in the sense of Alexandrov, further we say ``$\Alex[0]$ space''.
\item $X$ has no quadruples of type $\quadra(4)$ if and only if 
$X$ is a $\CAT[0]$ space which is also called Hadamard space
\end{itemize}

We will need one corollary of this statement.

\begin{thm}{Corollary}
Any 5-point metric space $\spc{X}$ admits a distance preserving embedding into $\RR^{4}$, $\RR^{3,1}$, or $\RR^{2,2}$.
Moreover, if $\spc{X}$ satisfies the $\CAT(0)$ comparison, then only the first two cases can appear; that is, $\spc{X}$ admits a distance preserving embedding into $\RR^{4}$ or $\RR^{3,1}$.
\end{thm}

\parit{Proof.}
\qeds

\begin{thm}{Lemma}
Suppose that quadruple of points $a,b,c,d\in \RR^{3,1}$ with induced distances form a $\CAT(0)$ metric space.
Assume the extensions of two opposide edges of $ab$ and $cd$ can be connected by a time-like curve.
Then if $(a,b,c,d)$ satisfy $\CAT(0)$ condition if one cannot connect two opposite sides by a time like-curve.
  
\end{thm}


\section{???}

Consider the space $\mathcal{A}$ of all $5$ point array in $\RR^3$ will be such that all 5 points do not lie on one plane and no three points lie on one line.
Note that the space $\mathcal{A}$ is connected.

The space $\mathcal{A}$ can be subdivided into the following 7 subsets $\mathcal{A}_{-3},\dots \mathcal{A}_{3}$
depending on the orientation of simplexes formed by 4 points.
More precisely, a $5$ point array  $(x_1,x_2,x_3,x_4,x_5)$ defines an affine map from a 4-simplex to $\RR^3$.
The 4-simplex has 5 faces, and each facet may be mapped in an orientation-preserving, degenerate, or orientation-reversing way.
Denote by $n_+$, $n_0$ and $n_-$ the number of orientation-preserving, degenerate, or orientation-reversing facets.
Since all 5 points would lie in one plane we have $n_0$ is at most 1, otherwise.
Therefore the value $m=n_--n_+$ can take an integer value between $-3$ and $3$, in this case we say that an array belong so $\mathcal{A}_m$.

Every two quadraliterals in the array have 3 common points that define a plane.
If the remaing points lie on the opposite sides from the plane then the corresponding facets have same orientation, if they lie on one side then the orientations are opposite.
Therefore the 7 subsets $\mathcal{A}_{-3},\dots \mathcal{A}_{3}$ can be described the following way:


$\mathcal{A}_{-3}$ --- a tetrahedron with reversed orientation and one point inside.

$\mathcal{A}_{-2}$ --- a tetrahedron with reversed orientation and one point on a facet.

$\mathcal{A}_{-1}$ --- double triangular pyramid formed by two tetrahedrons with reversed orientation.

$\mathcal{A}_{0}$ --- a pyramid over a convex quadraliteral 

$\mathcal{A}_{1}$ --- double triangular pyramid formed by two tetrahedrons with preserved orientation.

$\mathcal{A}_{2}$ --- a tetrahedron with preserve orientation and one point on a facet.

$\mathcal{A}_{3}$ --- a tetrahedron with preserved orientation and one point inside.

Note that the complement $\spc{A}\backslash \spc{A}_0$ has two connected components formed by $\mathcal{A}_{-}=\mathcal{A}_{-3}\cup \mathcal{A}_{-2}\cup\mathcal{A}_{-1}$ and $\mathcal{A}_{+}=\mathcal{A}_{3}\cup \mathcal{A}_{2}\cup\mathcal{A}_{1}$.
Observe that each array in $\mathcal{A}_{-}$ has at least 3 positively oriented facets and each array in $\mathcal{A}_{+}$ has at least 3 negatively oriented facets.


\section{Proof}


\begin{thm}{Key lemma}
Let $K$ a 4-simplex in $\RR^{3,1}$.
Suppose that the distances between the 5 vertexes of $K$ defines a metric that satisfies $\CAT(0)$ comparison.
Then we can choose a time-like orientation on $\RR^{3,1}$ so that $K$ has at most 2 upper facets with respect to any positive time-like direction.
\end{thm}

\parit{Proof.}
Choose a generic time like direction $v$, note that each of 5 factes of $K$ is either lower or upper with respect to $v$;
without loss of generality we can assume that it has at least 3 lower facets.
Since there is at least one upper facet, $K$ has 3 or 4 lower facets with respact to $v$.

Choose the time-like orientation on $\RR^{3,1}$ so that $v$ positive; that is, it points into the future.
Choose another generic time-like positive vector $w$.
Let us connect $v$ to $w$ by a \emph{generic} one-parameter family of time-like vectors $v_t$;
that is, for any time $t$ except finite number of values all facets of $K$ are either lower nor upper and at each exceptional time $t_i$ there is only one facet $F_i$ of $K$ that is \emph{vertical} with respect to $v_i$; that is, $F\parallel v_{t_i}$.

Note that at each exceptional times the number of lower triangles may be changed one of two ways $4\leftrightarrow 3$ or $1\leftrightarrow 2$.
Indeed ...

Since at the beginning $v_0=v$, we have 4 or 3 lower facets,
all the time we will keep having 4 or 3.
In particular $K$ has  4 or 3 lower facets with respect to~$w$.

Finally note that any 3 facets contain all the edges of $K$.
Since for any generic time-like positive vector $w$ at least 3 facets are lower,
every edge is lower with respect to $w$.
Whence every edge is lower with respect to any positive time-like vector $w$;
equivalently every edge lies in the lower side $\Sigma_-$ of the surface of $K$.

By \ref{lem:sides}, $\Sigma_-$ is $\CAT(0)$ length space.
Since every edge of $K$ lies in $\Sigma_-$, we have that 
\qeds

{\sloppy
\printbibliography[heading=bibintoc]
\fussy
}
\end{document}

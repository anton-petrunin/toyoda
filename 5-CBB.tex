\documentclass{article}
\usepackage{quad}

\hypersetup{pdftitle={5-point comparison},
pdfauthor={Nina Lebedeva and Anton Petrunin}}

\newcommand{\Addresses}{{\bigskip\footnotesize

\noindent Nina Lebedeva,
\par\nopagebreak
 \textsc{Saint Petersburg State University, 7/9 Universitetskaya nab., St. Petersburg, 199034, Russia}
\par
\nopagebreak
 \textsc{St. Petersburg Department of V.A. Steklov Institute of Mathematics of the Russian Academy of Sciences, 27 Fontanka nab., St. Petersburg, 191023, Russia}
  \par\nopagebreak
  \textit{Email}: \texttt{lebed@pdmi.ras.ru}

\medskip

\noindent   Anton Petrunin, 
\par\nopagebreak
 \textsc{Math. Dept. PSU, University Park, PA 16802, USA.}
  \par\nopagebreak
  \textit{Email}: \texttt{petrunin@math.psu.edu}
  
}}

\begin{document}
%\pagestyle{empty}


\title{5-point comparison}
\author{Nina Lebedeva and Anton Petrunin}

\date{}
\maketitle
\begin{abstract}
We give an if-and-only-if condition on a five-point metric spaces that admit embeddings into nonnegatively curved Riemannian manifolds.
\end{abstract}

\nofootnote{\textbf{Keywords:} CBB(0), finite metric space, comparison inequality, Alexandrov comparison, Lang--Schroeder--Sturm inequality.}
\nofootnote{\textbf{MSC:} 53C23, 30L15, 51F99.}

\section{Introduction}

Toponogov theorem provides an if-and-only-if condition on a metric on four-point space that adimits an isometric embedding into a nonnegatively curved Riemannian manifold.
The only-if part is the actual Toponogov theorem and the if part follows from a result of Abraham Wald \cite[\S 7]{wald}.

In this note we give an analogous condition for the five-point space.
Its only-if part is well known, but if part is new.
The formulation uses the so-called Lang--Schroeder--Sturm inequality \cite{lang-schroeder, sturm} which we are about to discuss.

Consider an $(n+1)$-point array $(p,x_1,\dots x_n)$ in a metric space $X$ with nonnegative curvature.
We say that the array satisfies Lang--Schroeder--Sturm inequality (briefly,  $\LLS(n)$)
if 
\[\sum_{i,j}a_{i,j}\cdot \lambda_i\cdot\lambda_j\ge 0\]
for any nonnegative values $\lambda_1,\dots,\lambda_n$;
here $a_{ij}=|p-x_i|_X^2+|p-x_j|_X^2-|x_i-x_j|_A^2$
and we denote by $|p-q|_X$ the distance between points in $X$.

Recall that any point array in a nonnegatively curved Alexandrov space (in particular in any complete nonnegatively curved Riemannian manifold) meets the Lang--Schroeder--Sturm inequality.
In particular, 
given a finite metric space $F$ the
Lang--Schroeder--Sturm inequalities for all relabeling of points in $F$
give a necessary condition for existence of isometric embedding of $F$ into an Alexandrov space with nonnegative curvature.
In this note we show that this condition is sufficient if $F$ has at most 5 points.

\begin{thm}{Theorem}\label{thm:main}
A five-point metric space $F$ admits an isometric embedding into a complete nonnegatively curved Riemannian manifold
if and only if all Lang--Schroeder--Sturm inequalities hold in $F$.
\end{thm}

Since Lang--Schroeder--Sturm inequalities provides a necessary condition,
to prove the theorem we need to construct a complete nonnegatively curved Riemannian manifold that contains an isometric copy of any given 5-point space that meets all Lang--Schroeder--Sturm inequalities.
We first construct an Alexandrov space with this property and then smooth it slightly keeping the required property.

Our theorem provides an affirmative answer to Question~6.2 in \cite{lebedeva-petrunin}.
For 6-point metric spaces a direct analog of the theorem does not hold,
but Question~6.3 in \cite{lebedeva-petrunin} might be an appropriate analog.

An analogous problem for 5-point sets in nonpositively curved spaces was solved by Tetsu Toyoda \cite{toyoda}; another solution is given in \cite{lebedeva-petrunin}.
The 6 point case is open;
Question 6.1 in \cite{lebedeva-petrunin} might contain an answer.

Our proof uses brute force search that was originally done on computer.
We want to thank Arseniy Akopyan and Alexander Gil who helped us with programming.
We present a hand-made proof that was found later.


\section{LLS(\textit{n}) and (\textit{n}+1)-comparison}

The (\textit{n}+1)-comparison is another condition that holds for any $(n+1)$-point array in a nonnegatively Alexandrov space $A$ \cite{alexander2019alexandrov}.
It says that given a point array $p,x_1,\dots,x_n$ in $A$ 
there is an array $\tilde p,\tilde x_1,\dots,\tilde x_n$ in a Hilbert space such that 
\[
|\tilde p-\tilde x_j|_{\HH}=|p-x_i|_{A}
\quad
\text{and}
\quad
|\tilde x_i-\tilde x_j|_{\HH}\ge |x_i-x_j|_{A}.
\]
for all $i$ and $j$.

For general metric spaces, the (\textit{n}+1)-comparison implies the $\LLS(n)$.
For $n\ge 5$ the converse does not hold \cite[Section 8]{lebedeva-petrunin-zolotov}.
In this section we will show that these two conditions are equivalent for $n\le 4$.

\begin{thm}{Claim}\label{clm:(4+1)=LSS(4)}
For any 5-point array $p,x_1,\dots,x_4$, the
$\LLS(4)$-inequality is equivalent to $(4+1)$-comparison.
\end{thm}

Applying the claim, we get the following reformulation of the main theorem.

\begin{thm}{Reformulation}\label{thm:main-(4+1)}
A five-point metric space $F$ admits an isometric embedding into an Alexandrov space with nonnegative curvature
if and only if it satisfies (4+1)-comparison for all relabelings.
\end{thm}

The following proof is nearly identical to the proof of \cite[Proposition 4.1]{lebedeva-petrunin-zolotov}.

\parit{Proof of \ref{clm:(4+1)=LSS(4)}.}
Suppose $p,x_1,\dots,x_4$ satisfies $\LLS(4)$;
we need to show that it also meets the (4+1)-comparison.
Choose a smooth function $\phi\:\RR\to \RR$ such that $\phi(x)=0$ is $x\ge0$ and $\phi(x)>0$, $\phi'(x)<0$ if $x<0$.
Consider a configuration of points $\tilde p,\tilde x_1,\dots,\tilde x_4\in \HH$ such that 
\[
|\tilde p-\tilde x_j|_{\HH}=|p-x_i|_{A}.
\]
and
\[
s=\sum_{i<j}\phi(|\tilde x_i-\tilde x_j|_{\HH}- |x_i-x_j|_{A})
\]
takes its minimal value.

Note that $s\ge0$;
if $s=0$, then we get the required configuration.

Suppose $s>0$.
Consider the graph $\Gamma$ with vertices 4 vertices labeled by $\tilde x_1,\tilde x_2,\tilde x_3,\tilde x_4$ such that 
$(\tilde x_i,\tilde x_j)$ is an edge if and only if $|\tilde x_i-\tilde x_j|_{\HH}<|x_i-x_j|_{A}$.
By triangle inequality, $\Gamma$ contains no end-vertices.
Therefore it is isomorphic to one of the following graphs.

\begin{figure}[ht!]
\centering
\includegraphics{mppics/pic-401}
\end{figure}

Further, we will assume that $\tilde p=0$.
Note that any point $\tilde x_i$ cannot lie in an open half-space with all its
adjacent points.
Indeed, if it would be the case, then we could slightly move this
vertex increasing the distances to all its adjacent vertexes and keeping the conditions on the other distances.

In the 6-edge case, $0$ lies in the convex hull of $\{\tilde x_1,\tilde x_2,\tilde x_3,\tilde x_4\}$.
In particular, we can write $0$ as a sum $\lambda_1\cdot \tilde x_1+\dots+\lambda_4\cdot\tilde x_4$ with $\lambda_i\ge0$.
The latter contradicts $\LLS(4)$.

Similarly, in the 5- and 3-edge cases, we can assume that $\tilde x_1\tilde x_2\tilde x_3$ is a 3-cycle of $\Gamma$.
In this case we can write $0$ as a sum $\lambda_1\cdot \tilde x_1+\lambda_2\cdot\tilde x_2+\lambda_3\cdot\tilde x_3$ with $\lambda_i\ge0$ and arrive at a contradiction.

Finally the 4-edge graph (that is, the 4-cycle) cannot occur.
In this case we may think that $\tilde x_1,\tilde x_2,\tilde x_3,\tilde x_4$ is the 4-cycle.
Note that the points $\tilde x_1,\tilde x_2,\tilde x_3,\tilde x_4$  lie in one plane so that the direction of $\tilde x_1$ is opposite to $\tilde x_3$,
and the direction of $\tilde x_2$ is opposite to $\tilde x_4$.
Let us think that this is the horizontal plane in $\RR^3$.
Then rotating the pair $\tilde x_1$, $\tilde x_3$ slightly up and 
the pair $\tilde x_2$, $\tilde x_4$  slightly down, decreases~$s$ --- a contradiction.
\qeds

\section{Associated form}

Let us recall a construction from \cite{petrunin-2017}.
Let $\bm{x}=(x_1,\dots,x_n)$ be a point array in a metric space.

Choose a simplex $\triangle$ in $\RR^{n-1}$; for example, we can take the standard simplex with the first $(n-1)$ of its vertices $v_1,\dots,v_n$ form the standard basis on $\RR^{n-1}$,
 and $v_n=0$.

Consider a quadratic form $W_{\bm{x}}$ on $\RR^{n-1}$ that is uniquely defined by
\[W_{\bm{x}}(v_i-v_j)=|x_i-x_j|^2_X\] 
for all $i$ and $j$.
The constructed quadratic form $W_{\bm{x}}$ will be called
the \emph{form associated to the point array $\bm{x}$}.
The following claim is self-evident:

\begin{thm}{Claim}\label{clm:W>=0}
An array $\bm{x}=(x_1,\dots,x_n)$ in a metric space $X$ is isometric to an array in a Euclidean space if and only if 
$W_{\bm{x}}(v)\ge 0$
for any $v\in \RR^{n-1}$.
\end{thm}


In particular, the condition $W_{\bm{x}}\ge 0$ for a triple $\bm{x}=(x_1,x_2,x_3)$ is equivalent to 
the three triangle inequalities for the distances between $x_1$, $x_2$, and $x_3$.
For an $n$-point array, it implies that $W_{\bm{x}}(v)\ge 0$ for any vector $v$ in a plane spanned by a triple of vertices of $\triangle$.

\parbf{Lang--Schroeder--Sturm inequalities.}
Consider an $n$-point array $\bm{x}$.
Assume $W_{\bm{x}}(w)<0$ for some $w\in\RR^{n-1}$.
Denote by $L_w$ the line spanned by $w$.
Consider the projection along $L_w$ from $\RR^{n-1}$ to the quotient space $\RR^{n-2}\z=\RR^{n-1}/L_w$.
The following claim is a reformulation Sturm--Lang--Shroeder inequalities for all relabeling of $\bm{x}$:

\begin{thm}{Claim}\label{clm:W(w)<0}
Let $\bm{x}=(x_1,\dots,x_n)$ be an $n$-point array in an Alexandrov space with nonnegative curvature.
Suppose that $W_{\bm{x}}(w)<0$ for some nonzero vector $w\in\RR^{n-1}$.
Then projection of each vertex of $\triangle$ to the quotient space $\RR^{n-1}/L_w$ does not lie in the projection of its opposite facet.
\end{thm}

If $W_{\bm{x}}(w)\le0$, then the same construction can be applied, but the projection of a vertex, say $v_1$, of $\triangle$ can lie in the interior of the projection of its opposite facet;
in this case we say that the array $\bm{x}$ is \emph{tense}.
The point $x_1$ in $\bm{x}$ that corresponds to $v_1$ is called \emph{marked point of} of the tense array.
For tense arrays we have the following rigidity statement:

\begin{thm}{Claim}\label{clm:W(w)==0}
Any tense array of an Alexandrov space $A$ with nonnegative curvature is isometric to a point array of a Euclidean space.
Moreover its marked point corresponds to a point in the relative interior of convex hull of the points in the Euclidean array.
\end{thm}

\parit{Proof.}
Given a point array $\bm{x}=(x_1,\dots,x_n)$ in $A$, denote by $v_1,\dots v_n$ the corresponding vertices of $\triangle$.

Suppose $\bm{x}=(x_1,\dots,x_n)$ is a tense array with marked point $x_1$;
that is, there is a vector $w$ is in the set $K_1$ of all linear combinations of $v_j-v_1$ with positive weights such that $W_{\bm{x}}(w)\le 0$. 

By \ref{clm:W(w)<0}, we have that $W_{\bm{x}}(v)\ge 0$ for any $v$ in the convex combination of $v_j-v_1$.
In particular $W_{\bm{x}}(w)=0$.
Since $K_1$ is open and $w\in K_1$, we get that $W_{\bm{x}}\ge 0$.
It remains to apply \ref{clm:W>=0}.
\qeds

\section{Extremal metrics}\label{sec:ext}

Denote by $\mathcal{A}_5$ the space of metrics on a 5-point set set $F=\{a,b,c,d,e\}$ that admits an embedding into an Alexandrov space with nonnegative curvature.
The associated quadratic forms for spaces in $\mathcal{A}_5$ form a convex cone in the space of all quadratic forms on $\RR^4$.
The latter follows since nonnegative curvature in the sense of Alexandrov survives after rescaling and passing to a product space.

Denote by $\mathcal{B}_5$ the space of metrics on $F$ that satisfies all Lang--Schroeder--Sturm inequalities for all relabelings.
As well as for $\mathcal{A}_5$, the associated forms for spaces in $\mathcal{B}_5$ form a convex cone in the space of all quadratic forms on $\RR^4$.

Since the associated quadratic form describes its metric completely, we may identify $\mathcal{A}_5$ and $\mathcal{B}_5$ with $\RR^{10}$ --- the space of quadratic forms on $\RR^4$.
This way we can think that $\mathcal{A}_5$ and $\mathcal{B}_5$ are convex cones in $\RR^{10}$.

The set $\mathcal{B}_5$ is a cone so it does not have extremal points except the origin.
The origin corresponds to degenerate metric with all zero distances.
But $\mathcal{B}_5$ is a cone over a convex compact set $\mathcal{B}_5'$ in the sphere $\mathbb{S}^9\subset \RR^{10}$.
The extremal points of $\mathcal{B}_5'$ correspond to extremal half-lines of $\mathcal{B}_5$;
metrics on extremal half-lines will be called \emph{extremal}.
Note that if an extremal metric $\rho$ lies in the interior of a line segment between metrics $\rho'$ and $\rho''$ in $\mathcal{B}_5$, then both metrics $\rho'$ and $\rho''$ are proportional to $\rho$.

Since Lang--Schroeder--Sturm inequalities are necessary for existance of isometric embedding into nonnegatively curved Alexandrov space,
we have that 
\[\mathcal{A}_5\subset\mathcal{B}_5.\]
To prove the theorem we need to show that the opposite inclusion holds as well.
Since $\mathcal{B}_5$ is the convex hull of its extremal metrics, it is sufficient to prove the following:

\begin{thm}{Proposition}\label{prop:main}
Given an extremal space $F$ in $\mathcal{B}_5$, there is a complete nonnegatively curved Riemannian manifold that contains an isometric copy of $F$.
\end{thm}

\parit{Proof.}
Note that any extremal space  $F$ contains a tense set.
If not then arbitrary slight change of metric keeps it in $\mathcal{B}_5$;
the latter is impossible for an extremal metric.

The remaining part of the proof is broken into cases:
\begin{itemize}
\item $F$ contains a 5-point  tense set; it follows from \ref{clm:W(w)==0}. In this case the needed Riemannian manifold is a Euclidean space.
\item $F$ contains a 4-point  tense set; it follows from in Proposition~\ref{prop:4-tense} below.
\item $F$ contains only 3-point  tense sets; it follows from in Proposition~\ref{prop:3-tense}. (This is the hardest part of the proof.) 
\qeds
\end{itemize}

\section{Four-point tense set}\label{sec:4-tense}

\begin{thm}{Proposition}\label{prop:4-tense}
Suppose that a 5-point metric space $F$ satisfies all Lang--Schroeder--Sturm inequalities and contains a 4-point tense set.
Then $F$ is isometric to a subset of complete nonnegatively curved Riemannian manifold. 
\end{thm}

In the following proof we first construct a nonnegatively curved Alexandrov space $W$ with isometric copy of $F$ and then smooth it.
The space $W$ will be a doubling of a convex polyhedral set in $\RR^3$.

\parit{Proof.}
Let us label the points in $F$ by $p$, $q$, $x_1$, $x_2$, and $x_3$ so that the array $(p,x_1,x_2,x_3)$ is tense with marked point $p$.

By \ref{clm:W(w)==0}, we can choose an array $(\tilde p, \tilde x_1, \tilde x_2, \tilde x_3)$ in $\RR^2$ that is isometric to  $(p, x_1, x_2, x_3)$.
Consider $\RR^2$ as a plane in $\RR^3$.

By \ref{clm:(4+1)=LSS(4)}, we can apply the (4+1)-comparison.
It implies the existence of point $\tilde q\in\RR^3$ such that
\begin{align*}
&&
|\tilde x_i-\tilde q|_{\RR^3}&\ge|x_i-q|_F,
&
|\tilde p-\tilde q|_{\RR^3}&=|p-q|_F
\intertext{for any $i$. 
Further, note that there are points $\tilde q_1$, $\tilde q_2$, $\tilde q_3$ in the plane that contains $\tilde p, \tilde x_1, \tilde x_2$, and $\tilde x_3$ such that}
|\tilde p-\tilde q_i|_{\RR^3}&\ge |p-q|_F,
&
|\tilde x_j-\tilde q_i|_{\RR^3}&\ge|x_j-q|_F,
&
|\tilde x_i-\tilde q_i|_{\RR^3}&=|x_i-q|_F,
\end{align*}
for all $i\ne j$.
The existence of such point follow from the following inequalities and equalities
\begin{align*}
\angk{x_i}{q}{x_j}+\angk{x_i}{q}{x_k}+\angk{x_i}{x_j}{x_k}&\le 2\cdot\pi,
&
\angk{x_i}{q}{x_j}&=\measuredangle\hinge{\tilde x_i}{\tilde q}{\tilde x_j},
\\
\angk{x_i}{q}{p}+\angk{x_i}{q}{x_j}+\angk{x_i}{p}{x_j}&\le 2\cdot\pi,
&
\angk{x_i}{q}{p}&=\measuredangle\hinge{\tilde x_i}{\tilde q}{\tilde p}.
\end{align*}
for any permutation $(i,j,k)$ of $(1,2,3)$.

Now let us show that there is a point $\tilde s\in\RR^2$ such that
\begin{align*}
|\tilde p-\tilde s|_{\RR^3}&\le|p-q|_F,
&
|\tilde x_i-\tilde s|_{\RR^3}&\le|x_i-q|_F.
\end{align*}
for all $i$.
In other words, we need to show that intersection of the following 4 closed balls have nonempty intersection: 
$\bar B[\tilde p,|p-q|_F]$ and
$\bar B[\tilde x_i,|x_i-q|_F]$ for all $i$.
Indeed, by the overlap lemma \cite{alexander2019alexandrov}, any 3 of these balls have nonempty intersection;
it remains to apply Helly's theorem.

The four perpendicular bisectors to 
$[\tilde s, \tilde q]$, 
$[\tilde s, \tilde q_1]$, 
$[\tilde s, \tilde q_2]$, 
$[\tilde s, \tilde q_3]$ cut from $\RR^3$ a closed convex set $V$ that contains $\tilde s$;
it might be a one-sided infinite triangular prism or, if $\tilde q$ lies in the plane of the triangle, a two-sided infinte quadrangular prism.
Note that $V$ contains the points $\tilde p$, $\tilde x_1$, $\tilde x_2$, $\tilde x_3$ as well.

Consider the doubling $W$ of $V$ with respect to its boundary;
denote by $\iota_1$ and $\iota_2$ the two isometric embedding $V\to W$.
By construction, the array $\hat p=(\iota_1(\tilde p)$, $\hat x_1=\iota_1(\tilde x_1)$, $\hat x_2=\iota_1(\tilde x_2)$, $\hat x_3=\iota_1(\tilde x_3)$,  $\hat s=\iota_2(\tilde s))$ in $W$ is isometric to the array $(p, q, x_1, x_2, x_3)$ in $F$.

Finally, we need to show that the obtained space can be smoothed into a Riemannian manifold that still has an isometric copy of $F$.
Indeed, all the minimizing geodesics between the 5-point subspace do not visit singular set of $W$ therefore slight change of metric in a neighborhood of the singular set will not create a problem.
We may stat with the subset $V'\subset V$ that lies on the distance $\delta>0$,
think of $V'$ lying in $\RR^4$, pass to its $\eps$-neighborhood.
The boundary of the obtained neighborhood is a convex hypersurface $W'$ in $\RR^4$.
For appropriately chousen small $\eps$ and $\delta$, it meets all our conditions except it is only $C^{1,1}$-smooth.
It is straightforward to smooth it so that the metric changes only near edges of $V$.
In this case so the set $F$ remains to be isometrically embedded.
\qeds


\section{Three-point tense sets}\label{sec:3-tense}

\begin{thm}{Proposition}\label{prop:3-tense}
Suppose that an extremal 5-point metric space $F$ contains only 3-point tense sets.
Then $F$ is isometric to a subset in a nonnegatively curved Riemannian manifold.
Moreover we can assume that $L$ is homeomorphic to a circle or the plane.
\end{thm}

A three-point tense set $\{a,b,c\}$ with marked point $b$ will be briefly denoted by $abc$.
Observe that $F$ has tense set $abc$ if and only if 
\[|a-b|_F+|b-c|_F\z=|a-c|_F.\]

\begin{wrapfigure}{r}{20mm}
\vskip-0mm
\centering
\includegraphics{mppics/pic-200}
\end{wrapfigure}

On the diagrams, we will mark three-point tense sets by connecting its points by a smooth curve so that the marked point is in the middle.
For example the given diagram describes a metric on $\{a,b,c,d,e\}$ with five tense sets $abc$, $bcd$, $cda$, $dae$, $aec$.

The proof of the proposition relies on the following classification:

\begin{thm}{Key lemma}\label{lem:key}
Let $F$ be an extremal 5-point metric space; suppose that it has no tense subsets with 4 and 5 points.
Then $F$ has one of three configurations of tense sets shown on the diagram.

In other words, the points in $F$ can be labeled by $\{a,b,c,d,e\}$ so that it has
one of the following three tense-set configurations:

\begin{wrapfigure}{r}{45mm}
\vskip3mm
\centering
\includegraphics{mppics/pic-202}
\end{wrapfigure}
\vskip-6mm
\begin{align*}
&abc, bcd, cde, dea, eab;
\\
&abc, bcd, cda, aec, bed;
\\
&abc, bcd, cda, dab, aec, bed.
\end{align*}

\end{thm}

In the following proof we use only small part of the classification in Lemma~\ref{lem:key},
but the proof of this \emph{small part} takes nearly as long as the complete classification.

\begin{wrapfigure}{r}{20mm}
\vskip-0mm
\centering
\includegraphics{mppics/pic-207}
\end{wrapfigure}

\parit{Proof of \ref{prop:3-tense} modulo \ref{lem:key}.}
Suppose that $F$ has tense configuration as on the diagram.
In other words, we can label points in $F$  by $\{x_1,x_2,x_3,x_4,x_5\}$ so that
\[|x_{i}-x_{i-1}|_F+|x_{i+1}-x_{i}|_F=|x_{i+1}-x_{i-1}|_F\]
for any $i\pmod 5$.
In this case $F$ is isometric to a 5-point subset in the circle of length 
$\ell=|x_1-x_2|_F+\dots+|x_4-x_5|_F+|x_5-x_1|_F$.

By the key lemma it remains to consider the case of metric that has two tense triples with common marked point.
In this case, we can relabel $F$ by $x,v_1,v_2,w_1,w_2$ so that $F$ has tense triples $v_1xv_2$ and $w_1xw_2$.

\begin{wrapfigure}{r}{25mm}
\vskip-4mm
\centering
\includegraphics{mppics/pic-209}
\end{wrapfigure}

First we construct an Alexandrov space $L$ --- a disc triangulated by four triangles with vertices $\hat x$, $\hat v_1$, $\hat v_2$, $\hat w_1$, $\hat w_2$ as shown on the diagram.
Each of four triangles has flat metric with at most one singular point;
in other words each triangle is a solid geodesic triangle in a cone.
The sides of the triangles are the same as in $F$.

Note that the metric on the obtained disc is completely determined by the 12 angles of triangles.
It remains to choose these angles in such a way that $L$ has nonnegative curvature and the map $\iota\:F\to L$ defined by $x\mapsto \hat x$, $v_i\mapsto \hat v_i$, $w_i\mapsto \hat w_i$ is distance-preserving.
Note that by constuction $\iota$ is distance-nonexpanding, therefore we only need to show that it is distance-noncontracting.

This part is divided into two steps.

\parit{Step 1.} In this step we describe three groups of conditions on these 12 angles, and show that together they guarantee that $L$ is has nonnegative curvature in the sense of Alexandrov and $\iota$ is distance-preserving.

First we need that the 12 angles of the triangles are at least as large as the corresponding model angles;
that is,
\[
\measuredangle \hinge {\hat x}{\hat v_i}{\hat w_j}\ge \angk{x}{v_i}{w_j}, 
\quad
\measuredangle \hinge {\hat v_i}{\hat x}{\hat w_j}\ge \angk{v_i}{x}{w_j},
\quad
\measuredangle \hinge {\hat w_j}{\hat x}{\hat v_i}\ge \angk{w_j}{x}{v_i},\leqno({*})
\]
for all $i$ and $j$.
Note that this condition implies that side of each edge of the triangulation is a convex set in $L$.
In particular, it implies tha $\iota$ preserves the distances between all pairs $(x,v_i)$, $(x,w_j)$, and $(v_i,w_j)$ for all $i$ and $j$.
Moreover, it implies that any minimizing geodesic in $L$ may cross an edge $xv_i$ or $xw_j$ at most once.

Further choose a three-edge path in the triangulation connecting $v_1$ to $v_2$ (or $w_1$ to $w_2$), say $v_1w_1xv_2$.
Consider the polygonal line in the plane $\tilde v_1\tilde w_1\tilde x\tilde v_2$ with the same angles and sides as in $L$ such that $\tilde v_1$ and $\tilde v_2$ lie on the opposite sides from the line $\tilde w_1\tilde x$.
Set 
\[\tilde Z(v_1w_1xv_2)\df |\tilde v_1-\tilde v_2|.\]
The next group of conditions has eight inequalities:
\[
\begin{aligned}
|v_1- v_2|&\le \tilde Z(v_1w_ixv_2),
&
|v_1- v_2|&\le \tilde Z(v_1xw_iv_2),
\\
|w_1- w_2|&\le \tilde Z(w_1v_ixw_2),
&
|w_1- w_2|&\le \tilde Z(w_1xv_iw_2).
\end{aligned}
\leqno({*}{*})
\]
for any $i$.
This condition implies that $\iota$ preserves the distance between two remaining pairs $(v_1,v_2)$ and $(w_1,w_2)$.
Indeed, suppose that a minimizing geodesic $\gamma$ from $\hat v_1$ to $\hat v_2$ is shorter than $|v_1-v_2|_F$.
Since two edges $\hat v_1\hat x$ and $\hat x\hat v_2$ have total length $|v_1-v_2|_F$,
we can assume that $\gamma$ runs in a pair of two adjacent triangles, say $[\hat v_1\hat x\hat w_1]$ and $[\hat v_2\hat x\hat w_1]$.
From above $\gamma$ cross the edge $\hat x\hat w_1$ once.
Denote by $\hat z_1$ and $\hat z_2$ the singular points in the triangles $[\hat v_1\hat x\hat w_1]$ and $[\hat v_2\hat x\hat w_1]$.
We have the following 4 options: 

\begin{longtable}{|c|l|}
 \hline
\begin{minipage}{40mm}
\vskip3mm
\centering
\includegraphics{mppics/pic-351}
\\ \ 
\end{minipage}
&
\begin{minipage}{70mm}

\ 

Both points $z_1$ and $z_2$ lie on the left side from $\gamma$. In this case the statement follows from two inequalities in $({*})$; namely
\[\measuredangle \hinge {\hat x}{\hat v_i}{\hat w_1}\ge \angk{x}{v_i}{w_1}\]
for both choices of $i$.

\ 

\end{minipage}
\\ 
\hline

\begin{minipage}{40mm}
\vskip3mm
\centering
\includegraphics{mppics/pic-352}
\\ \ 
\end{minipage}
&
\begin{minipage}{70mm}

\ 

Both points $z_1$ and $z_2$ lie on the right side from $\gamma$. In this case the statement follows from two inequalities in $({*})$; namely
\[\measuredangle \hinge {\hat w_1}{\hat v_i}{\hat x}\ge \angk{w_1}{v_i}{x}\]
for both choices of $i$.

\ 

\end{minipage}
\\ 
\hline

\begin{minipage}{40mm}
\vskip3mm
\centering
\includegraphics{mppics/pic-353}
\\ \ 
\end{minipage}
&
\begin{minipage}{70mm}

\ 

Points $z_1$ lies on the left side from $\gamma$, 
and $z_2$ lies on its right side. In this case the statement follows from two inequalities in $({*}{*})$; namely
\[|v_1- v_2|\le \tilde Z(v_1xw_1v_2).\]

\ 

\end{minipage}
\\ 
\hline

\begin{minipage}{40mm}
\vskip3mm
\centering
\includegraphics{mppics/pic-354}
\\ \ 
\end{minipage}
&
\begin{minipage}{70mm}

\ 

Points $z_1$ lies on the right side from $\gamma$, 
and $z_2$ lies on its left side. In this case the statement follows from two inequalities in $({*}{*})$; namely
\[|v_1- v_2|\le \tilde Z(v_1w_1xv_2).\]

\ 

\end{minipage}
\\ 
\hline
\end{longtable}

We need one more group of conditions:
\[
\measuredangle \hinge {\hat v_i}{\hat x}{\hat w_1}
+\measuredangle \hinge {\hat v_i}{\hat x}{\hat w_2}
\le \pi,
\quad
\measuredangle \hinge {\hat w_i}{\hat x}{\hat v_1}
+\measuredangle \hinge {\hat w_i}{\hat x}{\hat v_2}
\le \pi.
\leqno(\asterism)
\]
for any $i$.
These inequalities in $(\asterism)$ imply that $L$ has convex boundary, and therefore, it is an Alexandrov space with nonnegative curvature.

\parit{Step 2.}
In this step we show that one can choose the 12 angles so that they meet all the conditions $({*})$, $({*}{*})$, and $(\asterism)$.
This part is done by means of elementary geometry.

By \ref{clm:(4+1)=LSS(4)}, we can apply (4+1) comparison for the array $x$, $v_1$, $v_2$, $w_1$, $w_2$. This way we get points $\tilde x$, $\tilde v_1$, $\tilde v_2$, $\tilde w_1$, $\tilde w_2$ such that 
\begin{align*}
|\tilde x-\tilde v_i|_{\HH}&=|x-v_i|_{A},
&
|\tilde x-\tilde w_i|_{\HH}&=|x-w_i|_{A},
&
|\tilde v_i-\tilde w_j|_{\HH}&\ge |v_i-w_j|_{A},
\\
|\tilde v_1-\tilde v_2|_{\HH}&\ge |v_1-v_2|_{A},
&
|\tilde w_1-\tilde w_2|_{\HH}&\ge |w_1-w_2|_{A}.
\end{align*}
The triangle inequality implies equality in the last two inequalities;
that is, each of points $(\tilde v_1,\tilde x,\tilde v_2)$ and $(\tilde w_1,\tilde x,\tilde w_2)$ lies on one line.
In particular, the whole configuration lies in $\RR^2$.

Set 
\[\measuredangle \hinge {\hat x}{\hat v_i}{\hat w_j}
=
\measuredangle \hinge {\tilde x}{\tilde v_i}{\tilde w_j}\]
for all $i$ and $j$.
Evidently this choice meets four conditions in $({*})$.

We still need to choose the remaining 8 angles $\measuredangle \hinge {\hat v_i}{\hat w_j}{\hat x}$ and $\measuredangle \hinge {\hat w_j}{\hat v_i}{\hat x}$ for all $i$ and $j$.
To do this we will extend the configuration $\tilde x,\tilde v_1,\tilde v_2,\tilde w_1,\tilde w_2$ by 8 more points 
$\tilde v_{ij}$, $\tilde w_{ij}$ so that we can set 
\[
\measuredangle \hinge {\hat v_i}{\hat w_j}{\hat x}=\measuredangle \hinge {\tilde v_i}{\tilde w_{ij}}{\tilde x},
\qquad
\measuredangle \hinge {\hat w_j}{\hat v_i}{\hat x}=\measuredangle \hinge {\tilde w_j}{\tilde v_{ji}}{\tilde x}.
\]
We assume that $|\tilde v_i-\tilde w_{ij}|_{\RR^2}=|v_i-w_j|$ and $|\tilde w_i-\tilde v_{ij}|_{\RR^2}=|w_i-v_j|$ for all $i$ and~$j$.
The conditions $({*})$, and $({*}{*})$ will follow if we could choose the points so that
\begin{align*}
|\tilde x-\tilde v_{ij}|&\ge | x-v_j|,
&
|\tilde x-\tilde w_{ij}|&\ge | x-w_j|,
\\
|\tilde v_{j'}-\tilde v_{ij}|&\ge | v_{j'}-v_j|,
&
|\tilde w_{j'}-\tilde w_{ij}|&\ge |w_{j'}-w_j|,
\end{align*}
here we assume that $j'\ne j$, so $2'=1$ and $1'=2$.

\begin{wrapfigure}{r}{35mm}
\vskip-4mm
\centering
\includegraphics{mppics/pic-255}
\end{wrapfigure}

The needed points $\tilde w_{21}$ and $\tilde w_{22}$ can be chosen to be reflections of $\tilde w_{1}$ and $\tilde w_{2}$ respectively across a line $\ell$ that we are about to describe.
Suppose $\triangle \tilde s\tilde w_1 \tilde w_2$ is a model triangle for $\triangle v_2w_1w_2$ such that $\tilde s$ lies on the opposite side from $\tilde v_2$ with respect to the line $\tilde w_1\tilde w_2$.
Then $\ell$ is the perpendicular bisector of $[\tilde v_1 \tilde s]$.
Since $|\tilde w_i-\tilde v_1|\ge | w_i- v_1|=|\tilde w_i-\tilde s|$ the points $\tilde w_1$ and $\tilde w_2$ lie on the opposite side from $\tilde v_1$ with respect to $\ell$.
Whence the conditions on $\tilde w_{21}$ and $\tilde w_{22}$ follow.
In addition, by construction we get $(\asterism)$ at $\tilde v_2$.

Similarly we construct the remaining 6 points.

\parit{Final step.} It remains to modify $L$ into a a plane with smooth Riemannian metric.
First note that $L$ is a convex subset of a plane with at most 4 singular points.
Further the geodesics between the 5-point subset in $L$ do not visit these singular points.
Therefore slight smoothing around singularities does not crates a problem.
\qeds


\parit{Proof of \ref{lem:key}.}
Since $F$ is extreme, any pair of points of $F$ must lie in a tense set.
If not, then all $\LLS(4)$ inequalities will remain to hold true after slight change of the distance between the pair.
The latter contradicts that $F$ is extreme.

\begin{wrapfigure}{l}{60mm}
\vskip-0mm
\centering
\includegraphics{mppics/pic-203}
\label{page:COPY}
\end{wrapfigure}

Suppose that two tense triples sharing two points.
All possible 4 configurations are shown on the diagram; they will be refereed as $C$, $O$, $P$, and $Y$ respectively.
Observe that in the configurations $P$ and $Y$, the set $\{a,b,c,d\}$ with marked point $b$ must be tense.
That is, if $P$ or $Y$ appear in $F$, then $F$ has a 4-point tense set which contradicts the assumptions.
It follows that $P$ and $Y$ cannot appear in our configuration.

Each tense triple gives at most two linear restrictions on the quadratic form.
The space of quadratic forms on $\RR^4$ is 10-dimensional.
Therefore, in order to be on an extremal ray, the quadratic form has to have at least 9 linear restrictions.
Hence $F$ contains at least 5 tense triples.

The remaining part of the proof is a brute force search of all possible configurationsthat satisfy the conditions above.
This search is roughly sketched on the following diagram that needs some explantion.
\begin{figure}[ht!]
\centering
\includegraphics{mppics/pic-210}
\end{figure}
We start with a configuration with one triple marked by solid line.
Chose a pair not in any triple of configuration;
connect it by a dashed line and search for an extra triple with this pair inside.
Each time we need to check up to 9 triples that contain the pair --- 3 choices for extra points and 3 choices for marked point in the obtained triple.
Some of them make a $P$ or $Y$ configuration with an existing triple, so they cannot be added.
If some of them can be added, then we draw a new diagram connected by arrow and continue.
In many cases the symmetry might be used to reduce the number of cases.
 

If there are no free pair (these are \ref{a(bcd)e+bcd}, \ref{abcdead}, \ref{abcda+aec+bed}, and \ref{abcdab+aec+bed}),
then we need to check all triples,
but due to symmetry the number of triples can be reduced.

Once we done with classification, we need to find all configurations with at least 5 triples (these start with column 5)
such that each pair belongs to one of tense triples (those that have no dashed line).
So we are left with three cases \ref{abcdead}, \ref{abcda+aec+bed}, and \ref{abcdab+aec+bed} marked in bold;
it proves the lemma.

The following table describes procedures at each node on the diagram.
It uses the following notations.
If a candidate triple, say $abd$ violates $Y$ rule with an existing triple, say $abc$, then we write \xcancel{$abd$}$Yabc$.
Similarly, if a candidate triple, say $adb$ violates $P$ rule with an existing triple, say $abc$, then we write \xcancel{$adb$}$Pabc$.
Further, assume a candidate, say $dbe$, does not violates the rules and so it can be added.
Suppose that after adding this triple we get a new configuration, say \ref{abc+dbe};
in this case we write $dbe{\to}$\ref{abc+dbe}.
Note that the new configuration is relabeled arbitrarily.
In the cases \ref{a(bcd)e+bcd}, \ref{abcdead}, \ref{abcda+aec+bed}, and \ref{abcdab+aec+bed} we check all triples up to symmetry.
The used symmetries are marked in the third column.


\newcounter{foo}
\setcounter{foo}{0}
\newcommand{\myitem}{\refstepcounter{foo}\thefoo}

\begin{longtable}{|c|c|c|l|}
 \hline
\myitem\label{abc}
&
\begin{minipage}{20mm}
\vskip3mm
\centering
\includegraphics{mppics/pic-301}
\\ \ 
\end{minipage}
&
$a\leftrightarrow c$
& 
\begin{tabular}{ll}
\xcancel{$dba$}$Yabc$;&
$dbe{\to}$\ref{abc+dbe};\\
$dab{\to}$\ref{abcd};&
$deb{\to}$\ref{abc+dae};\\
\xcancel{$adb$}$Pabc$;&
$edb{\to}$\ref{abc+dae}.\\
\end{tabular}
\\ 
\hline

\myitem\label{abcd}
&
\begin{minipage}{20mm}
\vskip3mm
\centering
\includegraphics{mppics/pic-302}
\\ \ 
\end{minipage}
&
$b\leftrightarrow c$
& 
\begin{tabular}{ll}
\xcancel{$adb$}$Pabc$;&
$ade{\to}$\ref{abcd+ade};\\
\xcancel{$abd$}$Yabc$;&
$aed{\to}$\ref{abcd+aed};\\
$bad{\to}$\ref{abcda};&
$ead{\to}$\ref{abcd+ade}.\\
\end{tabular}
\\ 
\hline

\myitem\label{abc+dae}
&
\begin{minipage}{20mm}
\vskip3mm
\centering
\includegraphics{mppics/pic-303}
\\ \ 
\end{minipage}
&
& 
\begin{tabular}{lll}
\xcancel{$dba$}$Yabc$;&
\xcancel{$dbc$}$Yabc$;&
$dbe{\to}$\ref{abc+d(ab)e};\\
\xcancel{$dab$}$Ydae$;&
$dcb{\to}$\ref{abcd+ade};&
\xcancel{$deb$}$Pdae$;\\
\xcancel{$adb$}$Pabc$;&
\xcancel{$cdb$}$Pabc$;&
\xcancel{$edb$}$Pdae$.\\
\end{tabular}
\\ 
\hline

\myitem\label{abc+dbe}
&
\begin{minipage}{20mm}
\vskip3mm
\centering
\includegraphics{mppics/pic-304}
\\ \ 
\end{minipage}
&
$c\leftrightarrow e$
& 
\begin{tabular}{ll}
\xcancel{$adb$}$Pabc$;&
$adc{\to}$\ref{abc+d(ab)e};\\
\xcancel{$abd$}$Yabc$;&
\xcancel{$acd$}$Pabc$;\\
\xcancel{$bad$}$Pdbe$;&
\xcancel{$cad$}$Pabc$.\\
\end{tabular}
\\ 
\hline

\myitem\label{abcd+aed}
&
\begin{minipage}{20mm}
\vskip3mm
\centering
\includegraphics{mppics/pic-305}
\\ \ 
\end{minipage}
&
& 
\begin{tabular}{lll}
\xcancel{$bea$}$Yaed$;&
\xcancel{$bec$}$Pabc$;&
\xcancel{$bed$}$Yaed$;\\
$bae{\to}$\ref{abcdea};&
\xcancel{$bce$}$Ybcd$;&
\xcancel{$bde$}$Pbcd$;\\
\xcancel{$abe$}$Yabc$;&
\xcancel{$cbe$}$Yabc$;&
\xcancel{$dbe$}$Pbcd$.\\
\end{tabular}
\\ 
\hline

\myitem\label{abcda}
&
\begin{minipage}{20mm}
\vskip3mm
\centering
\includegraphics{mppics/pic-306}
\\ \ 
\end{minipage}
&
$b\leftrightarrow d$
& 
\begin{tabular}{ll}
$cea{\to}$\ref{abcda+aec};&
\xcancel{$ceb$}$Pabc$;\\
\xcancel{$cae$}$Pabc$;&
\xcancel{$cbe$}$Yabc$;\\
\xcancel{$ace$}$Pabc$;&
\xcancel{$bce$}$Ybcd$.\\
\end{tabular}
\\ 
\hline

\myitem\label{abcd+ade}
&
\begin{minipage}{20mm}
\vskip3mm
\centering
\includegraphics{mppics/pic-307}
\\ \ 
\end{minipage}
&
& 
\begin{tabular}{lll}
\xcancel{$cea$}$Pade$;&
\xcancel{$ceb$}$Pabc$;&
\xcancel{$ced$}$Pbcd$;\\
\xcancel{$cae$}$Pade$;&
\xcancel{$cbe$}$Yabc$;&
\xcancel{$cde$}$Yade$;\\
\xcancel{$ace$}$Pabc$;&
\xcancel{$bce$}$Ybcd$;&
\xcancel{$dce$}$Ybcd$.\\
\end{tabular}
\\ 
\hline

\myitem\label{abc+d(ab)e}
&
\begin{minipage}{20mm}
\vskip3mm
\centering
\includegraphics{mppics/pic-308}
\\ \ 
\end{minipage}
&
& 
\begin{tabular}{lll}
$cda{\to}$\ref{abcd+aec+bed};&
\xcancel{$cdb$}$Pabc$;&
\xcancel{$cde$}$Pdae$;\\
\xcancel{$cad$}$Pabc$;&
\xcancel{$cbd$}$Yabc$;&
\xcancel{$ced$}$Pdae$;\\
\xcancel{$acd$}$Pabc$;&
\xcancel{$bcd$}$Pdbe$;&
$ecd{\to}$\ref{a(bcd)e+bcd};\\
\end{tabular}
\\ 
\hline

\myitem\label{abcdea}
&
\begin{minipage}{20mm}
\vskip3mm
\centering
\includegraphics{mppics/pic-309}
\\ \ 
\end{minipage}
&
$c\leftrightarrow d$
& 
\begin{tabular}{ll}
\xcancel{$bea$}$Ydea$;&
\xcancel{$bec$}$Pabc$;\\
$bae{\to}$\ref{abcdead};&
\xcancel{$bce$}$Ybcd$;\\
\xcancel{$abe$}$Yabc$;&
\xcancel{$cbe$}$Yabc$.\\
\end{tabular}
\\ 
\hline

\myitem\label{abcda+aec}
&
\begin{minipage}{20mm}
\vskip3mm
\centering
\includegraphics{mppics/pic-310}
\\ \ 
\end{minipage}
&
& 
\begin{tabular}{lll}
\xcancel{$bea$}$Yaec$;&
\xcancel{$bec$}$Yaec$;&
$bed{\to}$\ref{abcda+aec+bed};\\
$bae{\to}$\ref{abcdaec};&
\xcancel{$bce$}$Ybcd$;&
\xcancel{$bde$}$Pbcd$;\\
\xcancel{$abe$}$Yabc$;&
\xcancel{$cbe$}$Yabc$;&
\xcancel{$dbe$}$Pbcd$.\\
\end{tabular}
\\ 
\hline

\myitem\label{abcd+aec+bed}
&
\begin{minipage}{20mm}
\vskip3mm
\centering
\includegraphics{mppics/pic-311}
\\ \ 
\end{minipage}
&
$b\leftrightarrow c$
& 
\begin{tabular}{ll}
\xcancel{$adb$}$Pbed$;&
\xcancel{$ade$}$Paec$;\\
\xcancel{$abd$}$Yabc$;&
\xcancel{$aed$}$Ybed$;\\
$bad{\to}$\ref{abcda+aec+bed};&
\xcancel{$ead$}$Pbed$.\\
\end{tabular}
\\ 
\hline

\myitem\label{a(bcd)e+bcd}
&
\begin{minipage}{20mm}
\vskip3mm
\centering
\includegraphics{mppics/pic-312}
\\ \ 
\end{minipage}
&
\begin{tabular}{l}
$a\leftrightarrow e$\\
$b\leftrightarrow d$
\end{tabular}
& 
\begin{tabular}{ll}
\xcancel{$abc$}$Pace$;&
\xcancel{$abd$}$Yabe$;
\\
\xcancel{$bca$}$Yace$;&
\xcancel{$bda$}$Yade$;
\\
\xcancel{$cab$}$Pbcd$;&
$dab{\to}$\ref{abcda+aec+bed}.
\\
\end{tabular}
\\ 
\hline

\myitem\label{abcdead}
&
\begin{minipage}{20mm}
\vskip3mm
\centering
\includegraphics{mppics/pic-313}
\\ \ 
\end{minipage}
&
$D_5$
& 
\begin{tabular}{l}
\xcancel{$dac$}$Pbcd$;\\
\xcancel{$acd$}$Ybcd$.\\
\end{tabular}
\\ 
\hline

\myitem\label{abcdaec}
&
\begin{minipage}{20mm}
\vskip3mm
\centering
\includegraphics{mppics/pic-314}
\\ \ 
\end{minipage}
&
& 
\begin{tabular}{lll}
\xcancel{$bea$}$Yaec$;&
\xcancel{$bec$}$Yaec$;&
\xcancel{$bed$}$Pdae$;\\
\xcancel{$bae$}$Ydae$;&
\xcancel{$bce$}$Ybcd$;&
\xcancel{$bde$}$Pdae$;\\
\xcancel{$abe$}$Yabc$;&
\xcancel{$cbe$}$Yabc$;&
\xcancel{$dbe$}$Pbcd$;\\
\end{tabular}
\\ 
\hline

\myitem\label{abcda+aec+bed}
&
\begin{minipage}{20mm}
\vskip3mm
\centering
\includegraphics{mppics/pic-315}
\\ \ 
\end{minipage}
&
$b\leftrightarrow d$
& 
\begin{tabular}{lll}
$bad{\to}$\ref{abcdab+aec+bed};&
\xcancel{$bec$}$Ybed$;&
\xcancel{$bea$}$Ybed$;
\\
\xcancel{$adb$}$Pbed$;&
\xcancel{$ecb$}$Ybcd$;&
\xcancel{$eab$}$Pbed$;
\\
\xcancel{$dba$}$Pbed$;&
\xcancel{$cbe$}$Yabc$;&
\xcancel{$abe$}$Yabc$;
\\
\end{tabular}
\\ 
\hline

\myitem\label{abcdab+aec+bed}
&
\begin{minipage}{20mm}
\vskip3mm
\centering
\includegraphics{mppics/pic-316}
\\ \ 
\end{minipage}
&
$D_4$
& 
\begin{tabular}{l}
\xcancel{$abe$}$Yabc$;\\
\xcancel{$bea$}$Pabc$.\\
\end{tabular}
\\ 
\hline
\end{longtable}
\qeds






{\sloppy
\printbibliography[heading=bibintoc]
\fussy
}

\Addresses
\end{document}

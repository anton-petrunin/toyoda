\documentclass{article}
\usepackage{quad}

\hypersetup{pdftitle={5-point comparison},
pdfauthor={Nina Lebedeva and Anton Petrunin}}

\newcommand{\Addresses}{{\bigskip\footnotesize

\noindent Nina Lebedeva,
\par\nopagebreak
 \textsc{Saint Petersburg State University, 7/9 Universitetskaya nab., St. Petersburg, 199034, Russia}
\par
\nopagebreak
 \textsc{St. Petersburg Department of V.A. Steklov Institute of Mathematics of the Russian Academy of Sciences, 27 Fontanka nab., St. Petersburg, 191023, Russia}
  \par\nopagebreak
  \textit{Email}: \texttt{lebed@pdmi.ras.ru}

\medskip

\noindent   Anton Petrunin, 
\par\nopagebreak
 \textsc{Math. Dept. PSU, University Park, PA 16802, USA.}
  \par\nopagebreak
  \textit{Email}: \texttt{petrunin@math.psu.edu}
  
}}

\begin{document}
%\pagestyle{empty}


\title{5-point comparison}
\author{Nina Lebedeva and Anton Petrunin}

\date{}
\maketitle
\begin{abstract}
We give a nesseary and sufficient condition on a five-point metric space that admits an embedding into an Alexadrov space with nonnegative curvature.
\end{abstract}

\nofootnote{\textbf{Keywords:} CBB(0), finite metric space, comparison inequality, Alexandrov comparison, Lang--Schroeder--Sturm inequality.}
\nofootnote{\textbf{MSC:} 53C23, 30L15, 51F99.}

\section{Introduction}

Let $A$ be an Alexandrov space with nonnegative curvature.
Consider an $(n+1)$-point array $p,x_1,\dots x_n$ in $A$.
Set $a_{ij}=|p-x_i|^2+|p-x_j|^2-|x_i-x_j|^2$.
Here we denote by $|p-q|$ the distance between points of a metric space.
Recall that the Lang--Schroeder--Sturm inequality, states that
\[\sum_{i,j}a_{i,j}\cdot \lambda_i\cdot\lambda_j\ge 0\]
for any nonnegative values $\lambda_1,\dots,\lambda_n$.

Suppose $F$ is a finite metric space; that is a finite set with a metric.
Note that Lang--Schroeder--Sturm inequalities for all relabeling of points in $F$
give a necessary condition for existence of isometric embedding of $F$ into an Alexandrov space with nonnegative curvature.
In this note we show that this condition is sufficient if $F$ has at most 5 points.

\begin{thm}{Theorem}\label{thm:main}
A five-point metric space $F$ admits an isometric embedding into an Alexandrov space with nonnegative curvature
if and only if all Lang--Schroeder--Sturm inequalities hold in $F$.
\end{thm}

For 6-point metric spaces the statement does not hold.
An example discussed in the appendix, where we also list a possible for 6-point case.
Spaces with 7 points and more look out of reach.

\section{Associated form}

In this section we recall some facts about the so-called \emph{associated} form introduced in \cite{petrunin-2017};
it is a quadratic form 
$W_{\bm{x}}$ on $\RR^{n-1}$ associated
to a given $n$-point array $\bm{x}\z=(x_1,\dots,x_n)$ in a metric space $X$.

\parbf{Construction.}
Choose a simplex $\triangle$ in $\RR^{n-1}$; for example, we can take the standard simplex with the first $(n-1)$ of its vertices $v_1,\dots,v_n$ form the standard basis on $\RR^{n-1}$,
 and $v_n=0$.

Recall that $|a-b|_X$ denotes the distance between points $a$ and $b$ in the metric space $X$.
Consider a quadratic form $W_{\bm{x}}$ on $\RR^{n-1}$ that is defined by
\[W_{\bm{x}}(v_i-v_j)=|x_i-x_j|^2_X\] 
for all $i$ and $j$.
The constructed quadratic form $W_{\bm{x}}$ will be called
the \emph{form associated to the point array $\bm{x}$}.
The following claim is self-evident:

\begin{thm}{Claim}\label{clm:W>=0}
An array $\bm{x}=(x_1,\dots,x_n)$ in a metric space $X$ is isometric to an array in a Euclidean space if and only if 
$W_{\bm{x}}(v)\ge 0$
for any $v\in \RR^{n-1}$.
\end{thm}


In particular, the condition $W_{\bm{x}}\ge 0$ for a triple $\bm{x}=(x_1,x_2,x_3)$ is equivalent to 
the three triangle inequalities for the distances between $x_1$, $x_2$, and $x_3$.
For an $n$-point array, it implies that $W_{\bm{x}}(v)\ge 0$ for any vector $v$ in a plane spanned by a triple $v_i,v_j,v_k$.
In particular, we get the following:

\begin{thm}{Observation}\label{triangle-inq}
Let $W_{\bm{x}}$ be a form on $\RR^{n-1}$ associated with a point array $\bm{x}\z=(x_1,\dots,x_n)$.
Suppose that $L$ is a subspace of $\RR^{n-1}$ such that
$W_{\bm{x}}(v)< 0$ for any nonzero vector $v\in L$.
Then the projections of any 3 vertices of $\triangle$ to the quotient space $\RR^{n-1}/L$ are not collinear.
\end{thm}

\parbf{Lang--Schroeder--Sturm inequalities.}
Consider an $n$-point array $\bm{x}$.
Assume $W_{\bm{x}}(w)<0$ for some $w\in\RR^{n-1}$.
Denote by $L_w$ the line spanned by $w$.
Consider the projection along $L_w$ from $\RR^{n-1}$ to the quotient space $\RR^{n-2}\z=\RR^{n-1}/L_w$.
The following claim is a reformulation Sturm--Lang--Shroeder inequalities for all relabeling of $\bm{x}$:

\begin{thm}{Claim}\label{clm:W(w)<0}
Let $\bm{x}$ be an $n$-point array in a complete length $\CBB(0)$ space.
Suppose that $W_{\bm{x}}(w)<0$ for some nonzero vector $w\in\RR^{n-1}$.
Then each projection of each $v_i$ to the quotient space $\RR^{n-1}/L_w$ is an extremal point of the projection of~$\triangle$.
\end{thm}

If $W_{\bm{x}}(w)\le0$, then the same construction can be applied.
In this case the projection of some vertex might lie in the convex hull of the remaining projections,
but we have the following rigidity statement:

\begin{thm}{Claim}\label{clm:W(w)==0}
Let $\bm{x}$ be an $n$-point array in a complete length $\CBB(0)$ space.
Suppose that $W_{\bm{x}}(w)= 0$ for some vector $w\in\RR^{n-1}$.
If the projection of $v_i$ in $\RR^{n-1}/L_w$ lies in the interior of projection of the simplex formed by $\set{v_j}{j\in S}$,
then the subset $\set{x_j}{j\in S\cup \{j\}}$ is isometric to a subset of Euclidean space.
Moreover the point $x_i$ corresponds to a point in the convex hull of the remaining points in the subset.
\end{thm}

\parit{Proof.}
By \ref{clm:W(w)<0}, we have that $W_{\bm{x}}(v)\ge 0$ for any $v$ in the convex combination of $v_j-v_i$ for $j\in S$.
Since $W_{\bm{x}}(w)=0$, we get that $W_{\bm{x}}(v)\ge 0$ for any $v$ in the subspace spanned by $v_j-v_i$ for $j\in S$.
It remains to apply \ref{clm:W>=0} for the array formed by $x_i$ and $\set{x_j}{j\in S}$.
\qeds

If $i$ and $S$ are as in the claim,
then the subset $T=\set{x_j}{j\in S\cup \{j\}}$ will be called \emph{tense set} of $X$ and the point $x_i$ will be called \emph{marked point of the tense set}; note that the marked point is uniquely defined by the set $T$.

\section{Extremal metrics}

Denote by $\mathcal{A}_5$ the space of metrics on a 5-point set set $F=\{a,b,c,d,e\}$ that admits an embedding into an Alexandrov space with nonnegative curvature.
The associated quadratic forms for spaces in $\mathcal{A}_5$ form a convex cone in the space of all quadratic forms on $\RR^4$.
The latter follows since nonnegative curvature in the sense of Alexandrov survives after rescaling and passing to a product space.

Denote by $\mathcal{B}_5$ the space of metrics on $F$ that satisfies all Lang--Schroeder--Sturm inequalities for all relabelings.
As well as for $\mathcal{B}_5$, the associated forms for spaces in $\mathcal{A}_5$ form a convex cone in the space of all quadratic forms on $\RR^4$.

Since the associated quadratic form describes its metric completely, we may identify $\mathcal{A}_5$ and $\mathcal{B}_5$ with the set in the space of $\RR^{10}$ --- the space of quadratic forms on $\RR^4$.
This way we can think that $\mathcal{A}_5$ and $\mathcal{B}_5$ are convex cones in $\RR^{10}$.

The set $\mathcal{B}_5$ is a cone so it does not have extremal points except the origin (the origin corresponds to degenerate metric with all zero distances).
But $\mathcal{B}_5$ is a cone over a convex compact set $\mathcal{B}_5'$ in the sphere $\mathbb{S}^9\subset \RR^{10}$.
The extremal points of $\mathcal{B}_5'$ correspond to extremal half-lines of $\mathcal{B}_5$.
If a metric $\rho$ on an extremal half-line can lie in the interior of a line segment between metrics $\rho'$ and $\rho''$ in $\mathcal{B}_5$, then both metrics $\rho'$ and $\rho''$ are proportional to $\rho$.
Such metrics (and the corresponding metric spaces) will be called \emph{extremal}.
Note that the following proposition implies the theorem.

Since Lang--Schroeder--Sturm inequalities are necessary for existance of isometric embedding into nonnegatively curved Alexandrov space,
we have that 
\[\mathcal{A}_5\subset\mathcal{B}_5.\]
In order to prove the theorem we need to show that the opposite inclusion holds as well.
Since $\mathcal{B}_5$ is the convex hull of its extremal metrics, it is sufficient to prove the following:

\begin{thm}{Proposition}\label{prop:main}
Given an extremal space $F$ in $\mathcal{B}_5$ there is an Alexandrov space that contains an isometric copy of $F$.
\end{thm}

Note that any extremal space  $F$ contain a tense set.
If not then arbitrary slight change of metric keeps it in $\mathcal{B}_5$;
the latter is impossible for an extremal metric.

The proof of proposition will be broken into cases:
\begin{itemize}
\item $F$ contains a 5-point  tense set; it follows from \ref{clm:W(w)==0}.
\item $F$ contains a 4-point  tense set; proved in Section~\ref{sec:4-tense}.
\item $F$ contains only 3-point  tense sets; proved in Section~\ref{sec:3-tense}.
\end{itemize}

\section{Four-point tense set}\label{sec:4-tense}

\begin{thm}{Proposition}
Suppose that a 5-point metric space $F$ satisfies all Lang--Schroeder--Sturm inequalities and contains a 4-point tense set.
Then $F$ is isometric to a subset of doubling of a convex polyhedral set in $\RR^3$;
in particular, $F$ admits an isometric embedding into a nonnegatively curved Alexandrov space.
\end{thm}

\parit{Proof.}
Let us label the points in $F$ by $p$, $q$, $x_1$, $x_2$, and $x_3$ so that the set $\{p,x_1,x_2,x_3\}$ is tense with marked point $p$.

By \ref{clm:W(w)==0}, we can choose an array $(\tilde p$, $\tilde x_1$, $\tilde x_2$, $\tilde x_3)$ in $\RR^3$ that is isometric to the array $(p$, $x_1$, $x_2$, $x_3)$ in $F$.
Let us construct points $\tilde q$, $\tilde q_1$, $\tilde q_2$, $\tilde q_3$, $\tilde s\in\RR^3$ such that
\begin{align*}
|\tilde p-\tilde q|&=|p-q|,
&
|\tilde x_i-\tilde q|&\ge|x_i-q|,
\\
|\tilde x_i-\tilde q_i|&=|x_i-q|,
&
|\tilde x_j-\tilde q_i|&\ge|x_j-q|,
&
|\tilde p-\tilde q_i|&\ge |p-q|.
\\
|\tilde p-\tilde s|&\le|p-q|,
&
|\tilde x_i-\tilde s|&\le|x_i-q|.
\end{align*}
for all $i\ne j$.
???

The four perpendicular bisectors to 
$[\tilde s, \tilde q]$, 
$[\tilde s, \tilde q_1]$, 
$[\tilde s, \tilde q_2]$, 
$[\tilde s, \tilde q_3]$ cut from $\RR^3$ a closed convex set $V$ that contains $\tilde s$.
Note that $V$ contains the points $\tilde p$, $\tilde x_1$, $\tilde x_2$, $\tilde x_3$ as well.
Consider the doubling $W$ of $V$ with respect to its boundary;
denote by $\iota_1$ and $\iota_2$ the two isometric embedding $V\to W$.
It remains to show that the array $(\iota_1(\tilde p)$, $\iota_1(\tilde x_1)$, $\iota_1(\tilde x_2)$, $\iota_1(\tilde x_3)$,  $\iota_2(\tilde s))$ in $W$ is isometric to the array $(p$, $q$, $x_1$, $x_2$, $x_3)$ n $F$.
???
\qeds


\section{Three-point tense sets}\label{sec:3-tense}

\begin{wrapfigure}{r}{20mm}
\vskip-4mm
\centering
\includegraphics{mppics/pic-200}
\end{wrapfigure}

A three-point tense set $\{a,b,c\}$ with marked point $b$ will be briefly denoted by $abc$.
Observe that $F$ has tense set $abc$ if and only if we have equality in the triangle inequality $|a-b|+|a-c|\z=|b-c|$.

On the diagrams, three-point tense sets will be marked by a smooth curve with marked point in the middle.
For example the diagram describes a metric on $\{a,b,c,d,e\}$ with five tense sets $abc$, $bcd$, $cda$, $dae$, $aec$.

\begin{thm}{Proposition}\label{prop:3-tense}
Suppose that an extremal 5-point metric space $F$ contains only 3-point tense sets.
Then $F$ is isometric to a subset in a nonnegatively curved Alexandrov space $L$.
Moreover we can assume that $L$ is homeomorphic to a circle or disc.
\end{thm}

The proof of the proposition relies on the following lemma:

\begin{thm}{Key lemma}\label{lem:key}
Let $F$ be an extremal 5-point metric space; suppose that it has no tense subsets with 4 or 5 elements.
Then $F$ has one of three configurations of tense sets shown on the diagram.

In other words, the points in $F$ can be labeled by $\{a,b,c,d,e\}$ so that it has
one of the following three tense-set configurations:

\begin{wrapfigure}{r}{45mm}
\vskip3mm
\centering
\includegraphics{mppics/pic-202}
\end{wrapfigure}
\vskip-6mm
\begin{align*}
&abc, bcd, cde, dea, eab;
\\
&abc, bcd, cda, aec, bed;
\\
&abc, bcd, cda, dab, aec, bed.
\end{align*}

\end{thm}

Let us first show that the lemma implies the remaining case in \ref{prop:main}.

\begin{wrapfigure}{r}{20mm}
\vskip-4mm
\centering
\includegraphics{mppics/pic-207}
\end{wrapfigure}

\parit{Proof of \ref{prop:3-tense} modulo \ref{lem:key}.}
Suppose that $F$ has tense configuration as on the diagram.
In other words, we can can label points in $F$  by $\{x_1,x_2,x_3,x_4,x_5\}$ so that
\[|x_{i}-x_{i-1}|+|x_{i+1}-x_{i}|=|x_{i+1}-x_{i-1}|\]
for any $i\pmod 5$.
In this case $F$ is isometric to a 5-point subset in the circle of length $|x_2-x_1|+\dots+|x_5-x_4|+|x_1-x_5|$.

By the key lemma it remains to consider the case of metric that has two tense triples with common marked point.
In this case, we can relabel $F$ so that these two tense triples are $v_1xv_2$ and $w_1xw_2$.

\begin{wrapfigure}{r}{25mm}
\vskip-4mm
\centering
\includegraphics{mppics/pic-209}
\end{wrapfigure}

The required Alexandrov space $L$ is homeomorphic to the disc;
it admits a triangluation into four triangles with vertices $\hat x$, $\hat v_1$, $\hat v_2$, $\hat w_1$, $\hat w_2$ as shown on the diagram.
Each of four triangles has flat metric with at most one singular point;
their sides are the same as in the model triangle and angles are at least as large as the corresponding model angle.

Note that the metric on the obtained disc is completely determined by the 12 angles of triangles.
It remains to choose these angles in such a way that the map $F\to L$ defined by $x\mapsto \hat x$, $v_i\mapsto \hat v_i$, $w_i\mapsto \hat w_i$ is distance preserving.
The latter holds if we have the have the following inequalities on angles
\[
\measuredangle \hinge {\hat x}{\hat v_i}{\hat w_j}\ge \angk{x}{v_i}{w_j}, 
\quad
\measuredangle \hinge {\hat v_i}{\hat x}{\hat w_j}\ge \angk{x}{v_i}{w_j},
\quad
\measuredangle \hinge {\hat w_j}{\hat x}{\hat v_i}\ge \angk{x}{v_i}{w_j},
\]
for all $i$ and $j$ and the following more complicated inequalities.
For a path from three edges in the triangulation connecting $v_1$ to $v_2$ (or $w_1$ to $w_2$), say $v_1w_1xv_2$.
Consider the polygonal line in the plane $\tilde v_1\tilde w_1\tilde x\tilde v_2$ with the same angles and sides such that $\tilde v_1$ and $\tilde v_2$ lie on the opposite sides from the line $\tilde w_1\tilde x$.
Set $\tilde Z(v_1w_1xv_2)\df |\tilde v_1-\tilde v_2|$.
Then we should have eight inequalities
\begin{align*}
|v_1- v_2|&\le \tilde Z(v_1w_1xv_2),
&
|v_1- v_2|&\le \tilde Z(v_1xw_1v_2),
\\
|v_1- v_2|&\le \tilde Z(v_1w_2xv_2),
&
|v_1- v_2|&\le \tilde Z(v_1xw_2v_2),
\\
|w_1- w_2|&\le \tilde Z(w_1v_1xw_2),
&
|w_1- w_2|&\le \tilde Z(w_1xv_1w_2),
\\
|w_1- w_2|&\le \tilde Z(w_1v_2xw_2),
&
|w_1- w_2|&\le \tilde Z(w_1xv_2w_2).
\end{align*}
In addition, to keep the disc convex, we need
\[
\measuredangle \hinge {\hat v_i}{\hat x}{\hat w_1}
+\measuredangle \hinge {\hat v_i}{\hat x}{\hat w_2}
\le pi,
\quad
\measuredangle \hinge {\hat w_i}{\hat x}{\hat v_1}
+\measuredangle \hinge {\hat w_i}{\hat x}{\hat v_2}
\le pi.
\]
for any $i$.
\qeds


\parit{Proof of \ref{lem:key}.}
Since $F$ is extreme, any pair of points of $F$ must lie in a tense set.
If not, then all (4+1)-comparisons will remain to hold true after slight change of the distance between the pair.
The latter contradicts that $F$ is extreme.

\begin{wrapfigure}{l}{60mm}
\vskip-0mm
\centering
\includegraphics{mppics/pic-203}
\end{wrapfigure}

Suppose that two tense triples sharing two points.
All possible 4 configurations are shown on the diagram; they will be refereed as $C$, $O$, $P$, and $Y$ respectively.
Observe that in the configurations $P$ and $Y$, the set $\{a,b,c,d\}$ with marked point $b$ must be tense.
Therefore $P$ and $Y$ cannot appear in $F$ since it has no 4-point tense sets.

Each tense triple gives at most two linear restrictions on the quadratic form.
The space of quadratic forms on $\RR^4$ is 10-dimensional.
Therefore, in order to be on an extremal ray, the quadratic form has to have at least 9 linear restrictions.
Hence $F$ contains at least 5 tense triples.

The remaining part of the proof is brute force search of all possible configurations;
it is sketched on the following diagram.
\begin{figure}[ht!]
\centering
\includegraphics{mppics/pic-210}
\end{figure}

We start with a configuration with one triple marked by solid line.
Chose a pair not in any triple of configuration;
connect it by a dashed line and search for an extra triple with this pair inside.
Each time we need to check up to 9 triples that contain the pair --- 3 choices for extra points and 3 choices for marked point in the obtained triple.
Some of them make a $P$ or $Y$ configuration with an existing triple, so they cannot be added.
If some of them can be added, then we draw a new diagram connected by arrow and continue.
In many cases the symmetry might be used to reduce the number of cases.
 

If there are no free pair (these are \ref{a(bcd)e+bcd}, \ref{abcdead}, \ref{abcda+aec+bed}, and \ref{abcdab+aec+bed}),
then we need to check all triples,
but due to symmetry the number of triples can be reduced.

Once we done with classification, we need to find all configurations with at least 5 triples (these start with column 5)
such that each pair belongs to one of tense triples (those that have no dashed line).
So we are left with three cases \ref{abcdead}, \ref{abcda+aec+bed}, and \ref{abcdab+aec+bed} marked in bold;
it proves the lemma.

The table describes procedures at each node on the diagram;
it use shortcut notation which we need to explain.
If a candidate triple, say $abd$ violates $Y$ rule with an existing triple, say $abc$, then we write \xcancel{$abd$}$Yabc$.
Similarly, if a candidate triple, say $adb$ violates $P$ rule with an existing triple, say $abc$, then we write \xcancel{$adb$}$Pabc$.
Further, assume a candidate, say $dbe$, does not violates the rules and so it can be added.
Suppose that after adding this triple we get a new configuration, say \ref{abc+dbe};
in this case we write $dbe{\to}$\ref{abc+dbe}.
In the cases \ref{a(bcd)e+bcd}, \ref{abcdead}, \ref{abcda+aec+bed}, and \ref{abcdab+aec+bed} we check all triples up to symmetry.

The used symmetries are marked in the third column.
Note that the new configuration is relabeled arbitrarily.


\medskip

\newcounter{foo}
\setcounter{foo}{0}
\newcommand{\myitem}{\refstepcounter{foo}\thefoo}

\begin{longtable}{|c|c|c|l|}
 \hline
\myitem\label{abc}
&
\begin{minipage}{20mm}
\vskip3mm
\centering
\includegraphics{mppics/pic-301}
\\ \ 
\end{minipage}
&
$a\leftrightarrow c$
& 
\begin{tabular}{ll}
\xcancel{$dba$}$Yabc$;&
$dbe{\to}$\ref{abc+dbe};\\
$dab{\to}$\ref{abcd};&
$deb{\to}$\ref{abc+dae};\\
\xcancel{$adb$}$Pabc$;&
$edb{\to}$\ref{abc+dae}.\\
\end{tabular}
\\ 
\hline

\myitem\label{abcd}
&
\begin{minipage}{20mm}
\vskip3mm
\centering
\includegraphics{mppics/pic-302}
\\ \ 
\end{minipage}
&
$b\leftrightarrow c$
& 
\begin{tabular}{ll}
\xcancel{$adb$}$Pabc$;&
$ade{\to}$\ref{abcd+ade};\\
\xcancel{$abd$}$Yabc$;&
$aed{\to}$\ref{abcd+aed};\\
$bad{\to}$\ref{abcda};&
$ead{\to}$\ref{abcd+ade}.\\
\end{tabular}
\\ 
\hline

\myitem\label{abc+dae}
&
\begin{minipage}{20mm}
\vskip3mm
\centering
\includegraphics{mppics/pic-303}
\\ \ 
\end{minipage}
&
& 
\begin{tabular}{lll}
\xcancel{$dba$}$Yabc$;&
\xcancel{$dbc$}$Yabc$;&
$dbe{\to}$\ref{abc+d(ab)e};\\
\xcancel{$dab$}$Ydae$;&
$dcb{\to}$\ref{abcd+ade};&
\xcancel{$deb$}$Pdae$;\\
\xcancel{$adb$}$Pabc$;&
\xcancel{$cdb$}$Pabc$;&
\xcancel{$edb$}$Pdae$.\\
\end{tabular}
\\ 
\hline

\myitem\label{abc+dbe}
&
\begin{minipage}{20mm}
\vskip3mm
\centering
\includegraphics{mppics/pic-304}
\\ \ 
\end{minipage}
&
$c\leftrightarrow e$
& 
\begin{tabular}{ll}
\xcancel{$adb$}$Pabc$;&
$adc{\to}$\ref{abc+d(ab)e};\\
\xcancel{$abd$}$Yabc$;&
\xcancel{$acd$}$Pabc$;\\
\xcancel{$bad$}$Pdbe$;&
\xcancel{$cad$}$Pabc$.\\
\end{tabular}
\\ 
\hline

\myitem\label{abcd+aed}
&
\begin{minipage}{20mm}
\vskip3mm
\centering
\includegraphics{mppics/pic-305}
\\ \ 
\end{minipage}
&
& 
\begin{tabular}{lll}
\xcancel{$bea$}$Yaed$;&
\xcancel{$bec$}$Pabc$;&
\xcancel{$bed$}$Yaed$;\\
$bae{\to}$\ref{abcdea};&
\xcancel{$bce$}$Ybcd$;&
\xcancel{$bde$}$Pbcd$;\\
\xcancel{$abe$}$Yabc$;&
\xcancel{$cbe$}$Yabc$;&
\xcancel{$dbe$}$Pbcd$.\\
\end{tabular}
\\ 
\hline

\myitem\label{abcda}
&
\begin{minipage}{20mm}
\vskip3mm
\centering
\includegraphics{mppics/pic-306}
\\ \ 
\end{minipage}
&
$b\leftrightarrow d$
& 
\begin{tabular}{ll}
$cea{\to}$\ref{abcda+aec};&
\xcancel{$ceb$}$Pabc$;\\
\xcancel{$cae$}$Pabc$;&
\xcancel{$cbe$}$Yabc$;\\
\xcancel{$ace$}$Pabc$;&
\xcancel{$bce$}$Ybcd$.\\
\end{tabular}
\\ 
\hline

\myitem\label{abcd+ade}
&
\begin{minipage}{20mm}
\vskip3mm
\centering
\includegraphics{mppics/pic-307}
\\ \ 
\end{minipage}
&
& 
\begin{tabular}{lll}
\xcancel{$cea$}$Pade$;&
\xcancel{$ceb$}$Pabc$;&
\xcancel{$ced$}$Pbcd$;\\
\xcancel{$cae$}$Pade$;&
\xcancel{$cbe$}$Yabc$;&
\xcancel{$cde$}$Yade$;\\
\xcancel{$ace$}$Pabc$;&
\xcancel{$bce$}$Ybcd$;&
\xcancel{$dce$}$Ybcd$.\\
\end{tabular}
\\ 
\hline

\myitem\label{abc+d(ab)e}
&
\begin{minipage}{20mm}
\vskip3mm
\centering
\includegraphics{mppics/pic-308}
\\ \ 
\end{minipage}
&
& 
\begin{tabular}{lll}
$cda{\to}$\ref{abcd+aec+bed};&
\xcancel{$cdb$}$Pabc$;&
\xcancel{$cde$}$Pdae$;\\
\xcancel{$cad$}$Pabc$;&
\xcancel{$cbd$}$Yabc$;&
\xcancel{$ced$}$Pdae$;\\
\xcancel{$acd$}$Pabc$;&
\xcancel{$bcd$}$Pdbe$;&
$ecd{\to}$\ref{a(bcd)e+bcd};\\
\end{tabular}
\\ 
\hline

\myitem\label{abcdea}
&
\begin{minipage}{20mm}
\vskip3mm
\centering
\includegraphics{mppics/pic-309}
\\ \ 
\end{minipage}
&
$c\leftrightarrow d$
& 
\begin{tabular}{ll}
\xcancel{$bea$}$Ydea$;&
\xcancel{$bec$}$Pabc$;\\
$bae{\to}$\ref{abcdead};&
\xcancel{$bce$}$Ybcd$;\\
\xcancel{$abe$}$Yabc$;&
\xcancel{$cbe$}$Yabc$.\\
\end{tabular}
\\ 
\hline

\myitem\label{abcda+aec}
&
\begin{minipage}{20mm}
\vskip3mm
\centering
\includegraphics{mppics/pic-310}
\\ \ 
\end{minipage}
&
& 
\begin{tabular}{lll}
\xcancel{$bea$}$Yaec$;&
\xcancel{$bec$}$Yaec$;&
$bed{\to}$\ref{abcda+aec+bed};\\
$bae{\to}$\ref{abcdaec};&
\xcancel{$bce$}$Ybcd$;&
\xcancel{$bde$}$Pbcd$;\\
\xcancel{$abe$}$Yabc$;&
\xcancel{$cbe$}$Yabc$;&
\xcancel{$dbe$}$Pbcd$.\\
\end{tabular}
\\ 
\hline

\myitem\label{abcd+aec+bed}
&
\begin{minipage}{20mm}
\vskip3mm
\centering
\includegraphics{mppics/pic-311}
\\ \ 
\end{minipage}
&
$b\leftrightarrow c$
& 
\begin{tabular}{ll}
\xcancel{$adb$}$Pbed$;&
\xcancel{$ade$}$Paec$;\\
\xcancel{$abd$}$Yabc$;&
\xcancel{$aed$}$Ybed$;\\
$bad{\to}$\ref{abcda+aec+bed};&
\xcancel{$ead$}$Pbed$.\\
\end{tabular}
\\ 
\hline

\myitem\label{a(bcd)e+bcd}
&
\begin{minipage}{20mm}
\vskip3mm
\centering
\includegraphics{mppics/pic-312}
\\ \ 
\end{minipage}
&
\begin{tabular}{l}
$a\leftrightarrow e$\\
$b\leftrightarrow d$
\end{tabular}
& 
\begin{tabular}{ll}
\xcancel{$abc$}$Pace$;&
\xcancel{$abd$}$Yabe$;
\\
\xcancel{$bca$}$Yace$;&
\xcancel{$bda$}$Yade$;
\\
\xcancel{$cab$}$Pbcd$;&
$dab{\to}$\ref{abcda+aec+bed}.
\\
\end{tabular}
\\ 
\hline

\myitem\label{abcdead}
&
\begin{minipage}{20mm}
\vskip3mm
\centering
\includegraphics{mppics/pic-313}
\\ \ 
\end{minipage}
&
$D_5$
& 
\begin{tabular}{l}
\xcancel{$dac$}$Pbcd$;\\
\xcancel{$acd$}$Ybcd$.\\
\end{tabular}
\\ 
\hline

\myitem\label{abcdaec}
&
\begin{minipage}{20mm}
\vskip3mm
\centering
\includegraphics{mppics/pic-314}
\\ \ 
\end{minipage}
&
& 
\begin{tabular}{lll}
\xcancel{$bea$}$Yaec$;&
\xcancel{$bec$}$Yaec$;&
\xcancel{$bed$}$Pdae$;\\
\xcancel{$bae$}$Ydae$;&
\xcancel{$bce$}$Ybcd$;&
\xcancel{$bde$}$Pdae$;\\
\xcancel{$abe$}$Yabc$;&
\xcancel{$cbe$}$Yabc$;&
\xcancel{$dbe$}$Pbcd$;\\
\end{tabular}
\\ 
\hline

\myitem\label{abcda+aec+bed}
&
\begin{minipage}{20mm}
\vskip3mm
\centering
\includegraphics{mppics/pic-315}
\\ \ 
\end{minipage}
&
$b\leftrightarrow d$
& 
\begin{tabular}{lll}
$bad{\to}$\ref{abcdab+aec+bed};&
\xcancel{$bec$}$Ybed$;&
\xcancel{$bea$}$Ybed$;
\\
\xcancel{$adb$}$Pbed$;&
\xcancel{$ecb$}$Ybcd$;&
\xcancel{$eab$}$Pbed$;
\\
\xcancel{$dba$}$Pbed$;&
\xcancel{$cbe$}$Yabc$;&
\xcancel{$abe$}$Yabc$;
\\
\end{tabular}
\\ 
\hline

\myitem\label{abcdab+aec+bed}
&
\begin{minipage}{20mm}
\vskip3mm
\centering
\includegraphics{mppics/pic-316}
\\ \ 
\end{minipage}
&
$D_4$
& 
\begin{tabular}{l}
\xcancel{$abe$}$Yabc$;\\
\xcancel{$bea$}$Pabc$.\\
\end{tabular}
\\ 
\hline
\end{longtable}
\qeds



{\sloppy
\printbibliography[heading=bibintoc]
\fussy
}

\Addresses
\end{document}
